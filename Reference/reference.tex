%
% The first command in your LaTeX source must be the \documentclass command.
\documentclass[acmsmall]{acmart}
% clear acmtog.cls footer settings
\fancyfoot{}
\setcopyright{none}
\renewcommand\footnotetextcopyrightpermission[1]{}
\pagestyle{plain}
%remove reference format
\settopmatter{printacmref=false}
\numberwithin{figure}{subsection}
%
\usepackage{hyperref}
\usepackage{listings}
\usepackage{fancyvrb}
\DefineVerbatimEnvironment{code}{Verbatim}{fontsize=\small}
\begin{document}

\title{A Totality Checker for a Dependently Typed Language}

%
% The "author" command and its associated commands are used to define the authors and their affiliations.
% Of note is the shared affiliation of the first two authors, and the "authornote" and "authornotemark" commands
% used to denote shared contribution to the research.
\author{Marty Stumpf}
\email{thealmartyblog@gmail.com}

%
% Keywords. The author(s) should pick words that accurately describe the work being
% presented. Separate the keywords with commas.
%\keywords{functional programming, dependent type, type checker, termination checking, totality checking}

%
%
% This command processes the author and affiliation and title information and builds
% the first part of the formatted document.
\maketitle
\thispagestyle{empty}
\tableofcontents
\clearpage
\section{Introduction}

This is the reference document for the
\href{https://github.com/thealmarty/totality-checker}{totality-checker}
repository, which is a totality checker for a dependently typed language
implemented in Haskell. The totality checker checks:

\begin{enumerate}
  \item strict positivity of constructors
  \item pattern matching coverage
  \item term termination
\end{enumerate}

To support these checks, the type checker has to support data type (hence the
strict positivity) and function (hence the pattern matching coverage)
declarations. I first describe the type checker without totality checks in
section \ref{sec:prelim}. Then I describe the mechanism for checking strict positivity
in section \ref{sec:spos}. After that, I describe the mechanism for checking
termination in section \ref{sec:termination}. Finally, I describe the mechanism
for checking the patterns of a function cover all cases in section
\ref{sec:pattern}.

\input{prelim.tex}
\section{Strict Positivity Checks}
\label{sec:spos}

This section concerns the property of the constructors of a data type declarations (see section \ref{sec:datatypes}). Following figure \ref{fig:datatype}'s notations, the strict positivity check ensures that the data type $D$ does not \emph{occur} (see figure \ref{fig:nonOccur}) in its own constructors ($c_i$). In other words, $D$ must only \emph{occur strictly positively} in $\Delta_i$.

The strict positivity condition rules out declarations that contain infinite loops. For example,

\begin{equation*}
  \begin{aligned}
    data &  & Bad : \star                 \\
         &  & bad : (Bad \to Bad) \to Bad \\
  \end{aligned}
\end{equation*}

The data type $Bad$ has one constructor, $bad$, it takes one argument, $\Delta_1 = (Bad \to Bad)$.

One can see from $\Delta_1$ that the data type $Bad$ is an argument to the function that is an input of its own constructor. Thus, there is a negative occurrence of $Bad$ in the type of the argument of the constructor. That is, $D$ does not occur strictly positively in $\Delta_1$. Non strictly-positive declarations are rejected because one can write a non-terminating function using them. Consider the following Haskell code using $Bad$ as defined above:

\begin{lstlisting}[language=haskell]
  getFun :: Bad -> (Bad -> Bad)
  getFun (bad f) = f
  
  omega :: Bad -> Bad
  omega f = (getFun f) f
  
  loop :: Bad
  loop = omega (bad omega)
\end{lstlisting}

$bad$ takes in a function $f$ with the type $Bad \to Bad$. Thus $bad \: f$ has type $Bad$. Because the return type of $(getFun \: f)$ is $(Bad \to Bad)$, $(getFun \: f) \: f$ has the return type of $Bad$. All these functions have proper types. Allowing non-strictly positive constructors allows $loop$, which infinitely loops.

In short, the data type to be defined cannot occur in its constructor's arguments' function domain or in an application.

\subsection{Positive Parameters}

We add a positivity tag to distinguish between strictly positive parameters and non-strictly positive parameters:

\begin{figure}[H]
  \begin{equation*}
    \begin{aligned}
      Pos &  & = &  & SPos  &  & \textrm{strictly positive}     \\
          &  & | &  & NSpos &  & \textrm{non-strictly positive} \\
    \end{aligned}
  \end{equation*}
\end{figure}

The set of positive parameter indices of a data type $D$ is represented by $\boldsymbol{pos(D)}$.

The information of whether a data type's parameters are strictly positive is stored in the signature for the check. The mapping for data type in the signature is now:

\begin{figure}[H]
  \begin{equation*}
    \begin{aligned}
      \Sigma D & : & (\textrm{value, Pos, integer}) \\
    \end{aligned}
  \end{equation*}
  \caption{Types of Data Type Mapping in a Signature}
\end{figure}

\subsection{Strictly Positive Occurrence}

To check that a data type occurs in its constructor arguments strictly positively, we only need to check its constructors that are functions. The data type cannot occur in constructors that don't take arguments. $sposConstructor$ checks that a constructor is strictly positive. It takes 4 arguments:

\begin{enumerate}
  \item the name of the data type being checked.
  \item the next fresh generic value.
  \item the list of positivity tag.
  \item the type of the constructor being checked.
\end{enumerate}

If the constructor is strictly positive, then $sposConstructor$ returns $unit$. Otherwise, an error occurs.

\begin{figure}[H]
  \begin{equation*}
    \begin{aligned}
      sposConstructor \: n \: k \: [pos] \: (Pi \: x \: v_A \: B^{\rho}) & = & spos \: k \: D \: v_A \: \land                                     \\
                                                                         &   & sposConstructor \: n \: (k+1) \: [pos] \: (eval \: B^{\rho:(x,k)}) \\
    \end{aligned}
  \end{equation*}
  \caption{Strict Positivity Test for a Constructor}
\end{figure}

$sposConstructor$ calls $\boldsymbol{spos}$, which checks for strictly positive occurrence, and takes 3 arguments:

\begin{enumerate}
  \item the next fresh generic value.
  \item $a$ the value to check the occurrence of.
  \item the value that $a$ may occur in.
\end{enumerate}

$spos$ returns a boolean: if the data type occurs strictly positively, it returns true. Otherwise, it returns false.

\begin{figure}[H]
  \begin{equation*}
    \begin{aligned}
      spos \: k \: D \: (Pi \: x \: v_A \: B^{\rho}) & = & nonOccur \: k \: D \: v_A \: \land                                             \\
                                                     &   & spos \: (k+1) \: D \: (eval \: B^{\rho:(x,k)})                                 \\
      spos \: k \: D \: (Lam \: x \: e^{\rho})       & = & spos \: (k+1) \: D \: (eval \: e^{\rho:(x,k)})                                 \\
      spos \: k \: D \: (D \: v_1 \dots v_m)         & = & nonOccur \: k \: D \: v_j, \forall j \in \{1 \dots m\}, \notin pos(D) \: \land \\
                                                     &   & spos \: c \: D \: v_j, \forall j \in pos(D)                                    \\
      spos \: k \: D \: (v \: v_1 \dots v_n)         & = & spos \: k \: D \: v \: \land                                                   \\
                                                     &   & nonOccur \: k \: D \: v_j \forall j \in \{1 \dots n\}                          \\
      spos \: k \: D \: a                            & = & True                                                                           \\
    \end{aligned}
  \end{equation*}
  \caption{Strictly Positive Occurrence}
\end{figure}

$spos$ calls $nonOccur$ which is a stronger requirement than $spos$. $nonOccur$ checks that $D$ does not occur in the term at all while $spos$ checks that $D$ occurs in the term strictly positively.

We need to check for non-occurrence in certain circumstances in to determine strict positivity. For example, for a constructor which takes a function as an argument, $D$ has to not occur in the argument of that function.
\section{Termination Checks}
\label{sec:termination}

The type checker as is catches invalid expressions (by type checking) and data declarations that enable non-terminating functions (by checking strict positivity). However, one can still declare functions that infinitely loop. For example, the function declaration

\begin{lstlisting}[language=haskell]
  foo :: Nat -> Nat
  foo x = foo x
\end{lstlisting}

passes the type checker, although $foo$ is not well defined.

The termination checker (see $\boldsymbol{Termination.hs}$) aims to alleviate the problem in two ways:

\begin{enumerate}
  \item Analyzing the syntax of the program using the size-change principle (section \ref{sec:synTermination}).
  \item Analyzing the type validity using \emph{sized types} (section \ref{sec:typeTermination}).
\end{enumerate}

\subsection{Syntactic Checks}
\label{sec:synTermination}

The size-change principle states that \emph{a program terminates on all inputs if every infinite call sequence would cause an infinite descent in some data values.}

In order to determine whether a program terminates, we need to know how the size of the \emph{semantical} values change during evaluation of a recursion. If the expression of the arguments from a recursive call ($e$) is \emph{semantically smaller} than the patterns on the right hand side ($p$), then it passes this part of the termination checker.

\subsubsection{Order}

An \emph{order} is a set that denotes the possible results when comparing an expression $e$ to a pattern $p$.

$e$ is either

\begin{enumerate}
  \item \emph{smaller than} $p$ ($\boldsymbol{<}$),
  \item \emph{smaller than or equal to} $p$ ($\boldsymbol{\leq}$),
  \item or its relation to $p$ is \emph{unknown} ($\boldsymbol{?}$).
\end{enumerate}

The order is determined by the following axioms:

\begin{enumerate}
  \item $x < c \: (p_1 \dots p_n), \: \forall x \in {p_1, \dots p_n}$ and when $c$ is an inductive constructor.
  \item $f \: (e_1 \dots e_n) \leq f$
\end{enumerate}

The call matrix:
%TODO
The rows represent the input parameters of the calling function and the columns the arguments of the called function.
For example, a $\boldsymbol{<}$ in cell $(i, j)$ of a call matrix expresses that the $j$'th argument of the call is strictly smaller than the $i$'th parameter of the calling function.
%TODO put call matrix in fig
\begin{figure}[H]
  \begin{equation*}
    \begin{aligned}
      f \: (p_{11} \dots p_{1n}) = e_1 \\
      \dots                            \\
      f \: (p_{m1} \dots p_{mn}) = e_m \\
    \end{aligned}
  \end{equation*}
  \caption{Function $f$ and its Call Matrix}
\end{figure}


\subsection{Type-based Checks}
\label{sec:typeTermination}

\section{Pattern Matching Coverage Checks}
\label{sec:pattern}

\end{document}
