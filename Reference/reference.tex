%
% The first command in your LaTeX source must be the \documentclass command.
\documentclass[acmsmall]{acmart}
% clear acmtog.cls footer settings
\fancyfoot{}
\setcopyright{none}
\renewcommand\footnotetextcopyrightpermission[1]{}
\pagestyle{plain}
%remove reference format
\settopmatter{printacmref=false}
%
\usepackage{hyperref}
\begin{document}

\title{Totality Checker for a Dependently Typed Language}

%
% The "author" command and its associated commands are used to define the authors and their affiliations.
% Of note is the shared affiliation of the first two authors, and the "authornote" and "authornotemark" commands
% used to denote shared contribution to the research.
\author{Marty Stumpf}
\email{thealmartyblog@gmail.com}

%
% Keywords. The author(s) should pick words that accurately describe the work being
% presented. Separate the keywords with commas.
%\keywords{functional programming, dependent type, type checker, termination checking, totality checking}

%
%
% This command processes the author and affiliation and title information and builds
% the first part of the formatted document.
\maketitle
\thispagestyle{empty}
\section{Introduction}

This is the reference document for the
\href{https://github.com/thealmarty/termination-checker}{termination-checker}
repository, which is a totality checker for a dependently typed language
implemented in Haskell. The totality checker checks for:

\begin{enumerate}
  \item strict positivity
  \item pattern matching coverage
  \item termination
\end{enumerate}

To support these checks, the type checker has to support data type (hence the
strict positivity) and function (hence the pattern matching coverage)
declarations. I first describe the type checker without totality checks in
section \ref{sec:prelim}. Then I describe the mechanism for checking strict positivity
in section \ref{sec:spos}. After that, I describe the mechanism for checking
termination in section \ref{sec:termination}. Finally, I describe the mechanism
for checking the patterns of a function cover all cases in section
\ref{sec:pattern}.

\section{Preliminaries (type checking without totality checks)}
\label{sec:prelim}
The type checker checks \textit{data type} and \textit{function} declarations. These declarations
consist of \textit{expressions}.

\subsection{Expressions}

An expression $e$ is one of the following (as in `Types.hs`):

\begin{equation*}
  \begin{aligned}
    e &  & = &  & \star           &  & \textrm{universe of small types} \\
      &  & | &  & x               &  & \textrm{variable}                \\
      &  & | &  & c               &  & \textrm{constructor name}        \\
      &  & | &  & D               &  & \textrm{data type name}          \\
      &  & | &  & f               &  & \textrm{function name}           \\
      &  & | &  & \lambda x.e     &  & \textrm{abstraction}             \\
      &  & | &  & (x:A) \to B     &  & \textrm{dependent function type} \\
      &  & | &  & e e_1 \dots e_n &  & \textrm{application}             \\
  \end{aligned}
\end{equation*}

\subsection{Evaluation and Values}

Because of dependent types, computation is required during type-checking. An
expression is \textit{evaluated} during type-checking to a \textit{value}. (See `Evaluator.hs`.)

Values may contain \textit{closures}. A closure \textbf{$e^{\rho}$} is a pair of an expression $e$ and an
\textit{environment} \textbf{$\rho$}.

Environments provide bindings for the free variables occurring in the
corresponding $e$.

\subsubsection{Values}

\begin{equation*}
  \begin{aligned}
    v &  & = &  & v \: (v_1 \dots v_n)     &  & \textrm{application}              \\
      &  & | &  & Lam \: x \: e^{\rho}     &  & \textrm{abstraction}              \\
      &  & | &  & Pi \: x \: v \: e^{\rho} &  & \textrm{dependent function space} \\
      &  & | &  & k                        &  & \textrm{generic value}            \\
      &  & | &  & \star                    &  & \textrm{universe of small types}  \\
  \end{aligned}
\end{equation*}

$v \: (v_1 \dots v_n)$, $Lam \: x \: e^{\rho}$, and $Pi \: x \: v \: e^{\rho}$
can be evaluated further (see more below) while $k$, $\star$, $c$, $f$, and $D$ are atomic values
which cannot be evaluated further. A generic value *k* represents the computed value of a variable during type
checking. More on this below.

The closures in $Lam \: x \: e^{\rho}$, and $Pi \: x \: v \: e^{\rho}$ do not have a
binding for $x$. If there is no concrete value, a fresh generic value $k$ would be
the binding for $x$ so that the closures can be evaluated.

\subsubsection{Evaluations}

An expression $e$ in environment $\rho$ are evaluated as follows (as in
Evaluator.hs):

\begin{equation*}
  \begin{aligned}
    eval \: (\lambda x . e)^{\rho}   & = & Lam \: x \: e^{\rho}                                            \\
    eval \: ((x:A) \to B)^{\rho}     & = & Pi \: x \: v_A \: B^{\rho}                                      \\
                                     &   & \textrm{ where } v_A = eval \: A^{\rho}                         \\
    eval \: (e e_1 \dots e_n)^{\rho} & = & app \: v \: v_1 \dots v_n                                       \\
                                     &   & \textrm{ where } v = eval \: e^{\rho}, v_i = eval \: e_i^{\rho} \\
    eval \: (\star)^{\rho}           & = & \star                                                           \\
    eval \: c^{\rho}                 & = & c                                                               \\
    eval \: f^{\rho}                 & = & f                                                               \\
    eval \: x^{\rho}                 & = & \textrm{ value of } x \textrm{ in } \rho                        \\
  \end{aligned}
\end{equation*}

Applications can be evaluated further as follows:

\begin{equation*}
  \begin{aligned}
    app \: u \: ()                                             & = & u                                                      \\
    app \: (u \: c_{11} \dots c_{1n}) \: (c_{21} \dots c_{2n}) & = & app \: u \: (c_{11} \dots c_{1n}, c_{21} \dots c_{2n}) \\
    app \: (Lam \: x \: e^{\rho}) \: (v,(v_1 \dots v_n))       & = & app \: v' \: (v_1 \dots v_n)                           \\
                                                               &   & \textrm{ where } v' = eval \: e^{\rho,x=v}             \\
    app \: f \: (v_1 ... v_n)                                  & = & app_{fun} \: f \: (v_1 \dots v_n)                      \\
                                                               &   & \textrm{ if } f \textrm{ is a function }               \\
    app \: v \: (v_1 \dots v_n)                                & = & v \: (v_1 \dots v_n)                                   \\
  \end{aligned}
\end{equation*}



\section{Strict Positivity Checks}
\label{sec:spos}

\section{Termination Checks}
\label{sec:termination}

\subsection{Syntactic Checks}

\subsection{Type-based Checks}

\section{Pattern Matching Coverage Checks}
\label{sec:pattern}
\end{document}
