%
% The first command in your LaTeX source must be the \documentclass command.
\documentclass[acmsmall]{acmart}
% clear acmtog.cls footer settings
\fancyfoot{}
\setcopyright{none}
\renewcommand\footnotetextcopyrightpermission[1]{}
\pagestyle{plain}
%remove reference format
\settopmatter{printacmref=false}
\numberwithin{figure}{subsection}
%
\usepackage{hyperref}
\usepackage{listings}
\usepackage{fancyvrb}
\DefineVerbatimEnvironment{code}{Verbatim}{fontsize=\small}
\begin{document}

\title{A Totality Checker for a Dependently Typed Language}

%
% The "author" command and its associated commands are used to define the authors and their affiliations.
% Of note is the shared affiliation of the first two authors, and the "authornote" and "authornotemark" commands
% used to denote shared contribution to the research.
\author{Marty Stumpf}
\email{thealmartyblog@gmail.com}

%
% Keywords. The author(s) should pick words that accurately describe the work being
% presented. Separate the keywords with commas.
%\keywords{functional programming, dependent type, type checker, termination checking, totality checking}

%
%
% This command processes the author and affiliation and title information and builds
% the first part of the formatted document.
\maketitle
\thispagestyle{empty}
\tableofcontents
\clearpage
\section{Introduction}

This is the reference document for the
\href{https://github.com/thealmarty/totality-checker}{totality-checker}
repository, which is a totality checker for a dependently typed language
implemented in Haskell. The totality checker checks:

\begin{enumerate}
  \item strict positivity of constructors
  \item pattern matching coverage
  \item term termination
\end{enumerate}

To support these checks, the type checker has to support data type (hence the
strict positivity) and function (hence the pattern matching coverage)
declarations. I first describe the type checker without totality checks in
section \ref{sec:prelim}. Then I describe the mechanism for checking strict positivity
in section \ref{sec:spos}. After that, I describe the mechanism for checking
termination in section \ref{sec:termination}. Finally, I describe the mechanism
for checking the patterns of a function cover all cases in section
\ref{sec:pattern}.

\section{type checking without totality checks}
\label{sec:prelim}
The type checker checks \emph{data type} and \emph{function} declarations. These declarations
consist of \emph{expressions}. Because of dependent types, we need to \emph{evaulate} expressions to \emph{values} during type checking. We also need a \emph{signature} that collects and carries information about the declarations during type checking.

\subsection{Expressions}

An expression $e$ is one of the following (as in $Types.hs$):

\begin{equation*}
  \begin{aligned}
    e &  & = &  & \star              &  & \textrm{universe of small types} \\
      &  & | &  & x                  &  & \textrm{variable (name)}         \\
      &  & | &  & c                  &  & \textrm{constructor (name)}      \\
      &  & | &  & D                  &  & \textrm{data type (name)}        \\
      &  & | &  & f                  &  & \textrm{function (name)}         \\
      &  & | &  & \lambda x.e        &  & \textrm{abstraction}             \\
      &  & | &  & (x:A) \to B        &  & \textrm{dependent function type} \\
      &  & | &  & e \: e_1 \dots e_n &  & \textrm{application}             \\
  \end{aligned}
\end{equation*}

\subsubsection{Functions}

A function declaration $f$ is a sequence of function \emph{clauses}.

A clause consists of a sequence of \textrm{patterns} $p$, which are
arguments to the function, and the right hand side expression $e$:

\begin{figure}[H]
  \begin{equation*}
    f \: (p_1 \dots p_n) = e
  \end{equation*}
  \caption{A Clause}
\end{figure}

A pattern $p$ is either a variable pattern, a constructor pattern (which
consists of patterns), or an inaccessible pattern:

\begin{figure}[H]
  \begin{equation*}
    \begin{aligned}
      p &  & = &  & x                    &  & \textrm{variable pattern}     \\
        &  & | &  & c \: (p_1 \dots p_n) &  & \textrm{constructor pattern}  \\
        &  & | &  & \underline{e}        &  & \textrm{inaccessible pattern} \\
    \end{aligned}
  \end{equation*}
  \caption{Patterns}
\end{figure}

\subsection{Values}

Because of dependent types, computation is required during type-checking. An
expression $e$ in an environment $\rho$ is \emph{evaluated} during type-checking to a type or a \emph{value} $v$.

Environments provide bindings for the free variables occurring in the
corresponding $e$. Thus, an environment $\rho$ is a list of pairs of variable and its type. Values of abstractions, and dependent functions contain \emph{closures}. A closure $\boldsymbol{e^{\rho}}$ is a pair of an expression $e$ and an
\emph{environment} $\boldsymbol{\rho}$.

\begin{figure}[H]
  \begin{equation*}
    \label{eq:values}
    \begin{aligned}
      v &  & = &  & v \: (v_1 \dots v_n)     &  & \textrm{application}              \\
        &  & | &  & Lam \: x \: e^{\rho}     &  & \textrm{abstraction}              \\
        &  & | &  & Pi \: x \: v \: e^{\rho} &  & \textrm{dependent function space} \\
        &  & | &  & k                        &  & \textrm{generic value}            \\
        &  & | &  & \star                    &  & \textrm{universe of small types}  \\
        &  & | &  & c                        &  & \textrm{constructor}              \\
        &  & | &  & f                        &  & \textrm{function}                 \\
        &  & | &  & D                        &  & \textrm{data type}                \\
    \end{aligned}
  \end{equation*}
  \caption{Values}
\end{figure}

$v \: (v_1 \dots v_n)$, $Lam \: x \: e^{\rho}$, and $Pi \: x \: v \: e^{\rho}$
can be evaluated further (see section \ref{sec:eval}) while $k$, $\star$, $c$, $f$, and $D$ are \emph{atomic values}
which cannot be evaluated further. A generic value $k$ represents the value of a variable during type
checking. $k$ is an integer. It is also used to collect inaccessible patterns during type checking (see section \ref{sec:typeCheckPattern}).

\subsection{Signature}
A signature $\Sigma$ carries information about all declared constants (constructors, data types, and functions). Each declaration adds newly defined constants to the signature, which is empty at the beginning. Each declaration that is added to the signature is a mapping of the constant to its information, including the evaluated value (type) of the constant. Specifically, a mapping in the signature maps:

\begin{itemize}
  \item a \textbf{function} to
        \begin{enumerate}
          \item its type
          \item its clauses
          \item a boolean indicating whether the clauses have been type-checked,
        \end{enumerate}
  \item a \textbf{constructor} to its type,
  \item a \textbf{data type} to its type and the number of parameters.
\end{itemize}

\begin{figure}[H]
  \begin{equation*}
    \begin{aligned}
      \Sigma f & : & (\textrm{value, a sequence of clauses, boolean}) \\
      \Sigma c & : & \textrm{value}                                   \\
      \Sigma D & : & (\textrm{value, integer})                        \\
    \end{aligned}
  \end{equation*}
  \caption{Types of Mappings in a Signature}
\end{figure}

The signature is in a \emph{state} monad so that we can retrieve the signature during type and totality checking. The mapping of a constant may be added before the constant is type-checked. For a function constant, the mapping contains the information about whether the clauses have been type-checked.

\subsection{Evaluations}
\label{sec:eval}

The $eval$ function evaluates an expression $e$ in environment $\rho$ to a
$value$ as follows (see $Evaluator.hs$):

\begin{figure}[H]
  \begin{equation*}
    \begin{aligned}
      eval \: (\lambda x . e)^{\rho}   & = & Lam \: x \: e^{\rho}                                            \\
      eval \: ((x:A) \to B)^{\rho}     & = & Pi \: x \: v_A \: B^{\rho}                                      \\
                                       &   & \textrm{ where } v_A = eval \: A^{\rho}                         \\
      eval \: (e e_1 \dots e_n)^{\rho} & = & app \: v \: v_1 \dots v_n                                       \\
                                       &   & \textrm{ where } v = eval \: e^{\rho}, v_i = eval \: e_i^{\rho} \\
      eval \: (\star)^{\rho}           & = & \star                                                           \\
      eval \: c^{\rho}                 & = & c                                                               \\
      eval \: f^{\rho}                 & = & f                                                               \\
      eval \: x^{\rho}                 & = & \textrm{ value of } x \textrm{ in } \rho                        \\
    \end{aligned}
  \end{equation*}
  \caption{Evaluation of Expressions}
\end{figure}

The closures in $Lam \: x \: e^{\rho}$, and $Pi \: x \: v \: e^{\rho}$ do not have a
binding for $x$. If there is no concrete value, a fresh generic value $k$ would be
the binding for $x$ so that the closures can be evaluated.

When possible, we perform $\beta$-reduction and inductive function application. The $app$ function takes in a $value$ as the first argument and a sequence of
$value$s as the second argument, and returns a $value$:

\begin{figure}[H]
  \begin{equation*}
    \begin{aligned}
      app \: u \: ()                                             & = & u                                                      \\
      app \: (u \: c_{11} \dots c_{1n}) \: (c_{21} \dots c_{2n}) & = & app \: u \: (c_{11} \dots c_{1n}, c_{21} \dots c_{2n}) \\
      app \: (Lam \: x \: e^{\rho}) \: (v,(v_1 \dots v_n))       & = & app \: v' \: (v_1 \dots v_n)                           \\
                                                                 &   & \textrm{ where } v' = eval \: e^{\rho:(x,v)}           \\
      app \: f \: (v_1 ... v_n)                                  & = & app_{fun} \: f \: (v_1 \dots v_n)                      \\
                                                                 &   & \textrm{ if } f \textrm{ is a function }               \\
    \end{aligned}
  \end{equation*}
  \caption{$\beta$-Reduction and Inductive Function Application}
\end{figure}

Regarding $app_{fun}$: when the first argument of $app$ is a function, we need to type-checked each
clause of the function and thus pattern matching is required before we can evaluate $app$ further.

\subsubsection{Pattern Matching and $app_{fun}$}

To perform inductive function application, we need to find a clause that matches all patterns. We match all clauses
of the function until we find the clause. To match the clauses, we need to match all patterns of
each clause.

If all patterns of a clause match against the argument values (checked by $match$ and $matchList$), then
the right hand side can be evaluated (by $match_{clause}$).

The sequence of clauses is then matched against the evaluated right hand side values (by $match_{cls}$).

The function $match$ takes three arguments:

\begin{enumerate}
  \item an environment $\rho$
  \item a pattern $p$
  \item a value $v$
\end{enumerate}

When the pattern is \textbf{inaccessible}, it returns the input environment.

When the pattern is a \textbf{variable}, it returns an environment that binds the variables in the patterns to values ($x$ to $v$) .

When the pattern is a \textbf{constructor}, it returns the input environment if the constructor name(s) match(s) the input value.

\begin{figure}[H]
  \begin{equation*}
    \begin{aligned}
      match \: \rho \: \underline{e} \: v                               & = & \rho                                                 \\
      match \: \rho \: x \: v                                           & = & \rho \textrm{ with } (x, v) \textrm{ added.}         \\
      match \: \rho \: (c \: ()) \: c                                   & = & \rho                                                 \\
      match \: \rho \: (c \: (p_1 \dots p_n)) \: (c \: (v_1 \dots v_n)) & = & matchList \: \rho (p_1 \dots p_n) \: (v_1 \dots v_n) \\
    \end{aligned}
  \end{equation*}
  \textrm{ where }
  \begin{equation*}
    \begin{aligned}
      matchList \: \rho \: () \: ()                        & = & \rho                                                                               \\
      marchList \: \rho \: (p_1 \dots) \: e \: (v_1 \dots) & = & matchList \: \rho' \:(p_2 \dots) \: (v_2 \dots)                                    \\
                                                           &   & \textrm{ where } \rho'                               = match \: \rho \: p_1 \: v_1 \\
    \end{aligned}
  \end{equation*}
  \caption{Pattern Matching}
\end{figure}

Pattern matching can fail and no environment is returned in such case. $match$ is called by $match_{clause}$, which evaluates and returns the type of the right hand side expression of a given clause.

The function $match_{clause}$ takes in four arguments:

\begin{enumerate}
  \item an environment $\rho$, which contains the variables in the matched patterns.
  \item a sequence of patterns $(p_1 \dots p_n)$
  \item the right hand side expression $e$
  \item a sequence of values $(v_1 \dots v_n)$
\end{enumerate}

It only returns a $value$ if $match$ returns an environment:

\begin{figure}[H]
  \begin{equation*}
    \begin{aligned}
      match_{clause} \: \rho \: () \: e \: (v_1 \dots v_n)      & = & app \: v \: (v_1 \dots v_n)                                \\
                                                                &   & \textrm{ where } v = eval \: e^{\rho}                      \\
      match_{clause} \: \rho \: (p_1 \dots) \: e \: (v_1 \dots) & = & match_{clause} \: \rho' \: (p_2 \dots) \: e \: (v_2 \dots) \\
                                                                &   & \textrm{ where } \rho' = match \: \rho \: p_1 \: v_1       \\
    \end{aligned}
  \end{equation*}
  \caption{Matching All Patterns of a Single Clause}
\end{figure}

If all the patterns are matched in a clause, then it returns the $value$ of the application with the right hand side $e$ evaluated. If any of the patterns doesn't match, nothing is returned.

$match_{cls}$ takes a sequence of clauses and a sequence of values as input and returns the matched clause's value:

\begin{figure}[H]
  \begin{equation*}
    \begin{aligned}
      match_{cls} \: ((f \: (p_1 \dots p_n) = e), clause_2, \dots) \: (v_1 \dots v_n) & = & v                                                               \\
      \textrm{ where } v                                                              & = & match_{clause} \: () \: (p_1 \dots p_n) \: e \: (v_1 \dots v_n) \\
      match_{cls} \: ((f \: (p_1 \dots p_n) = e), clause_2, \dots) \: (v_1 \dots v_n) & = & match_{cls} \: (clause_2 \dots) \: (v_1 \dots v_n)              \\
    \end{aligned}
  \end{equation*}
  \caption{Matching All Clauses}
\end{figure}

Each clause is tried until one is matched or there are no clauses left. If there are no clauses left to match, then we failed to find a clause that matches all the patterns and nothing is returned. If nothing is returned, $app_{fun}$ cannot be reduced further. If a value is returned, and all clauses are type-checked (see section \ref{sec:typeCheckPattern}), then we can apply the function. We reduce $app_{fun}$ as follows:

\begin{figure}[H]
  \begin{equation*}
    \begin{aligned}
      app_{fun} \: f \: (v_1 \dots v_n) & =             & v                                                               \\
                                        & \textrm{ if } & \Sigma f = (t, (clause_1 \dots clause_n), True)                 \\
                                        & \land         & v = match_{cls} \: (clause_1 \dots clause_n) \: (v_1 \dots v_n) \\
    \end{aligned}
  \end{equation*}
  \caption{$app_{fun}$}
\end{figure}

\subsection{Type Checking of Expressions}

Recall that declarations consist of expressions. Type checking a declaration therefore comprises type checking expressions. See $CheckExpr.hs$.

We use the bidirectional type checking technique. The type-checker has two modes:

\begin{enumerate}
  \item $checkExpr$ checks that the input expression $e$ is of type $v$.
  \item $inferExpr$ infers the type of the input expression.
\end{enumerate}

$checkExpr$ calls $inferExpr$ as needed to infer the type of the input expression and check that it is \emph{equal} to the input type. $eqVal$ defines the equality of two values.

$checkExpr$, $inferExpr$, and $eqVal$ are simultanously defined in the next sections.

All type checking functions return a state transformer monad $StateT \: Signature \: IO \: a$. The state is the signature, which contains information for type checking of data type/function declarations. The inner monad is the $IO$ monad.

$a$ is $unit$ for $checkExpr$ and $eqVal$. If no error occurs, each step of the $IO$ returns $\boldsymbol{()}$. $a$ is $value$ for $inferExpr$, because $inferExpr$ returns the type of the input expression.

\subsubsection{$checkExpr$}
\label{sec:checkExpr}

$checkExpr$ takes 5 arguments:

\begin{enumerate}
  \item $k$, the next generic value.
  \item environment $\rho$, which is a list of pairs of $n$ (variable name) and $k$ that binds a fresh generic value to a variable.
  \item environment $\Gamma$, which is a list of pairs of $n$ and $v$ that binds the type to a variable.
  \item expression $e$, the expression being checked.
  \item value $v$, the supposed type of the expression.
\end{enumerate}

If the expression is not of type $v$, an error occurs. Unless an error occurs, it returns a $unit$.

\begin{figure}[H]
  \begin{equation*}
    \begin{aligned}
      checkExpr \: k \: \rho \: \Gamma \: (\lambda \: x.e) \: v  & = & checkExpr \: (k + 1) \: \rho' \: \Gamma' \:e \: v'          \\
                                                                 &   & \textrm{ where } v = Pi \: y \: v_A \: t^{\rho}             \\
                                                                 &   & \rho' = \rho : (x, k),                                      \\
                                                                 &   & \Gamma' = \Gamma : (x, v_A),                                \\
                                                                 &   & v' = eval \: t^{\rho:(y, k)}                                \\
      checkExpr \: k \: \rho \: \Gamma \: ((x:A) \to B) \: \star & = & checkExpr \: k \: \rho \: \Gamma A \star              \land \\
                                                                 &   & checkExpr \: (k+1) \: \rho' \: \Gamma' \: B \: \star        \\
                                                                 &   & \textrm{ where } \rho' = \rho : (x, k),                     \\
                                                                 &   & \Gamma' = \Gamma:(x, eval \: A^{\rho}).                     \\
      checkExpr \: k \: \rho \: \Gamma \: e \: v                 & = & eqVal \: k \: (inferExpr \: k \: \rho \: \Gamma \: e) \: v  \\
    \end{aligned}
  \end{equation*}
  \caption{Checking Expressions, $checkExpr$}
\end{figure}

\subsubsection{$inferExpr$}

$inferExpr$ takes 4 arguments, which is the same first 4 arguments taken in by $checkExpr$. Unless an error occurs, it returns the inferred type of the input expression.

\begin{figure}[H]
  \begin{equation*}
    \begin{aligned}
      inferExpr \: k \: \rho \: \Gamma \: (e_1 \: e_2)                & = & eval \: B^{\rho:(x,eval \: e_2^{\rho})}                                                        \\
                                                                      &   & \textrm{ where } inferExpr \: k \: \rho \: \Gamma \: e_1 = Pi \: x \: v_A \: B^{\rho} \: \land \\
                                                                      &   & checkExpr \: k \: \rho \: \Gamma \: e_2 \: v_A                                                 \\
      inferExpr \: k \: \rho \: \Gamma \: e_1 \: (e_2 \: (e_3 \dots)) & = & v                                                                                              \\
                                                                      &   & \textrm{ where } inferExpr \: k \: \rho \: \Gamma \: (e_1 \: e_2) \: (e_3 \dots) = v           \\
      inferExpr \: k \: \rho \: \Gamma \: x                           & = & \textrm{value of x in } \Gamma                                                                 \\
      inferExpr \: k \: \rho \: \Gamma \: f                           & = & v                                                                                              \\
                                                                      &   & \textrm{ where } \Sigma \: f = (v,\_,\_) \textrm{ in the Signature.}                           \\
      inferExpr \: k \: \rho \: \Gamma \: c                           & = & v                                                                                              \\
                                                                      &   & \textrm{ where } \Sigma \: c = v \textrm{ in the Signature.}                                   \\
      inferExpr \: k \: \rho \: \Gamma \: D                           & = & v                                                                                              \\
                                                                      &   & \textrm{ where } \Sigma \: D = (v,\_) \textrm{ in the Signature.}                              \\
    \end{aligned}
  \end{equation*}
  \caption{Inferring Expressions, $InferExpr$}
\end{figure}

\subsubsection{$eqVal$}

To check if values $v_1$ and $v_2$ are equal, $eqVal$ takes 3 arguments:

\begin{enumerate}
  \item $k$, the generic value being examined.
  \item $v_1$
  \item $v_2$
\end{enumerate}

If $v_1$ does not equal to $v_2$, an error occurs. Unless an error occurs, it returns a unit.

\begin{figure}[H]
  \begin{equation*}
    \begin{aligned}
      eqVal \: k \: (v \: (v_1 \dots v_n)) \: (w \: (w_1 \dots w_n))                       & = & eqVal \: k \:v \: w \: \land                                                                     \\
                                                                                           &   & eqVal \: k \: v_j \: w_j, \: \forall j \in \: \{1 \dots n\}                                      \\
      eqVal \: k \: (Pi \: x_1 \: v_1 \: b_1^{\rho}) \: (Pi \: x_2 \: v_2 \: b_2^{\Gamma}) & = & eqVal \: k \: v_1 \: v_2 \: \land                                                                \\
                                                                                           &   & eqVal \: (k+1) \: (eval \: b_1^{\rho \: : \: (x_1,k)}) \: (eval \: b_2^{\Gamma \: : \: (x_2,k)}) \\
      eqVal \: k \: (Lam \: x_1 \: e_1^{\rho}) \: (Lam \: x_2 \: e_2^{\Gamma})             & = & eqVal \: (k+1) \: (eval \: e_1^{\rho \: : \: (x_1,k)}) \: (eval \: e_2^{\Gamma \: : \: (x_2,k)}) \\
      eqVal \: k \: v_1 \: v_2, v_1,v_2 \in \textrm{atomic values}                         & = & v_1 = v_2                                                                                        \\
    \end{aligned}
  \end{equation*}
  \caption{Equality of Values, $eqVal$}
\end{figure}

\subsection{Type Checking of Data Types}
\label{sec:datatypes}

In this section I describe how data type declarations are type checked. See \emph{$checkDataType.hs$}.

A data type $\boldsymbol{D}$, parameterised by parameters $\boldsymbol{p_1 \dots p_n}$, indexed over $\boldsymbol{\Theta}$, inductively defined by the constructors $\boldsymbol{c_1 \dots c_n}$ with the given types, is declared as follows:

\begin{figure}[H]
  \centering
  \begin{equation*}
    \begin{aligned}
      data &  & \: D \: (p_1:P_1) \dots (p_n:P_n) : \Theta \to \star       \\
           &  & c_1 : \Delta_1 \to D \: p_1 \dots p_n \: t_1^1 \dots t_m^1 \\
           &  & \dots                                                      \\
           &  & c_k : \Delta_k \to D \: p_1 \dots p_n \: t_1^k \dots t_m^k \\                                                           \\
    \end{aligned}
  \end{equation*}
  \caption{A Data Type Declaration}
  \label{fig:datatype}
\end{figure}

$D$'s constructor $c_i$ takes arguments ($\Delta_i$) and has the return type of $D \: p_1 \dots p_n \: t_1^i \dots t_m^i$ where $t_1^i \dots t_m^i \in \Theta$, the arguments of the index of $D$.

The parameters and their types, $(p_1:P_1) \dots (p_n:P_n)$, are stored in a \emph{telescope}, $\boldsymbol{\tau}$:
\begin{equation*}
  \tau = (p_1:P_1) \dots (p_n:P_n)
\end{equation*}


For example, the data types of $\boldsymbol{Nat}$, $\boldsymbol{List}$, and $\boldsymbol{Vec}$:

\textbf{The data type of \emph{natural numbers}} can be introduced by:

\begin{equation*}
  \begin{aligned}
    data &       & Nat : \star \\
         & zero: & Nat         \\
         & suc:  & Nat \to Nat \\
  \end{aligned}
\end{equation*}

$Nat$ is a data type of type $\star$. It has no parameter (there is no $p$). It is not indexed ($\Theta$ is empty).

Its telescope of its parameter ($\tau$) is empty.

$zero$ is a constructor ($c_1$) of $Nat$. It doesn't take any argument ($\Delta_1$ is empty). Its type is $Nat$.

$suc$ is another constructor ($c_2$) of $Nat$. It takes a $Nat$ as an argument ($\Delta_2$ = $Nat$), which is fine because $Nat$ is of type $\star$. Its return type is $Nat$.

\textbf{The data type family of \emph{lists}} is parameterised by $A$, without being indexed:

\begin{equation*}
  \begin{aligned}
    data &       & List \: (A:\star) : \star     \\
         & nil:  & List \: A                     \\
         & cons: & A \to List \: A \to List \: A \\
  \end{aligned}
\end{equation*}

$List$ is a data type of type $\star$. It has one parameter, $p_1 = A$. The type of the parameter is $\star$, i.e., $P_1 = \star$.

It is not indexed ($\Theta$ is empty).

Its telescope is $\tau = (A:\star)$.

$nil$ is a constructor ($c_1$) of $List$. It doesn't take any argument ($\Delta_1$ is empty). Its type is $List \: A$.

$cons$ is another constructor ($c_2$) of $List$. It takes two arguments, $\Delta_2 = A, List \: A$. Its return type is $List \: A$.

\textbf{The data type family of \emph{vectors}} is parameterised by $A$, indexed by the length of the list:

\begin{equation*}
  \begin{aligned}
    data &        & Vec \: (A:\star) : Nat \to \star               \\
         & vnil:  & Vec \: A \: zero                               \\
         & vcons: & A \to Vec \: A \: n \to Vec \: A \: (suc \: n) \\
  \end{aligned}
\end{equation*}

$Vec$ is a data type of type $\star$. It has one parameter, $p_1 = A$. The type of the parameter is $\star$, i.e., $P_1 = \star$.

It is indexed by a $Nat$, i.e., $\Theta = Nat$.

Its telescope is $\tau = (A : \star)$.

$vnil$ is a constructor ($c_1$) of $Vec$. It doesn't take any argument ($\Delta_1$ is empty). Its type is $Vec \: A \: zero$, i.e., $t_1^1 = zero$.

$vcons$ is another constructor ($c_2$) of $Vec$. It takes 2 arguments, $\Delta_2 = A, Vec \: A \: n$. Its return type is $Vec \: A \: (suc \: n)$, i.e., $t^2_1 = suc \: n$.

For a data type to be type-checked, the following conditions must hold. The functions in brackets are the respective functions checking the conditions:

\begin{enumerate}
  \item The data type is type-checked, ($\boldsymbol{checkDataType}$):
  \item All constructors are type-checked ($\boldsymbol{typeCheckConstructor}$). It's checked by calling the following functions:
        \begin{itemize}
          \item  $\boldsymbol{checkConType}$: Checks that the constructor arguments (in $\Delta_i$) have type $\star$, and the type of the constructor is of type $\star$ (the same type as the data type).
          \item  $\boldsymbol{checkTarget}$: Checks that the parameters in the result type of every constructor exactly match those of the telescope $\tau$.
        \end{itemize}
\end{enumerate}

In $\boldsymbol{typeCheckConstructor}$, one of the steps calls the function $\boldsymbol{sposConstructor}$, which checks that the constructors are strictly positive. See section \ref{sec:spos} for details.

\subsubsection{$checkDataType$}

$checkDataType$ makes sure the data type declaration is valid by checking the following conditions:

\begin{enumerate}
  \item The parameters have valid types.
  \item The index arguments $\Theta$ are of type $\star$.
  \item The data type is of type $\star$.
\end{enumerate}

(2) and (3) are required to ensure that one cannot declare a data type that infinite loop.

$checkDataType$ takes 5 arguments:

\begin{enumerate}
  \item $k$, the next generic value.
  \item environment $\rho$, which is a list of pairs of $n$ (variable name) and $k$ that binds a fresh generic value to a variable.
  \item environment $\Gamma$, which is a list of pairs of $n$ and $v$ that binds the type to a variable.
  \item the length of the telescope or the number of parameters.
  \item expression $e$, the expression being checked.
\end{enumerate}

Unless an error occurring, $checkDataType$ returns a $unit$.

\begin{figure}[H]
  \begin{equation*}
    \begin{aligned}
      checkDataType \: k \: \rho \: \Gamma \: n \: ((x:A) \to B) & = &
      \begin{cases}
        A \in Type & \text{if } k < n    \\
        A : \star  & \text{if } k \geq n \\
      \end{cases}                                                                                                                                 \land \\
                                                                 &   & \: checkDataType \: (k+1) \: \rho : (x,k) \: \Gamma : (x, (eval A^{\rho})) \: n \: B            \\
      checkDataType \: k \: \rho \: \Gamma \: n \: \star         & = & ()                                                                                              \\
    \end{aligned}
  \end{equation*}
  \caption{Type-Check Data Type, $checkDataType$}
\end{figure}

\subsubsection{$typeCheckConstructor$}

$typeCheckConstructor$ calls $checkConType$, which is very similar to $checkDataType$. It takes in the same arguments, but it checks the constructors instead. The pseudocode is ommitted here.

When we type-check a data type declaration (in $TypeChecker.hs$), both $checkDataType$ and $checkConType$ are called starting with $k=0$ and empty environments. They check all expressions.

\subsection{Type Checking of Functions}

In this section I describe how function declarations are type checked. See \emph{$checkFunction.hs$} and \emph{$Pattern.hs$}.

Recall that a function $f$ has one or more clauses, each clause contains pattern(s) $p$ and the right hand side $e$:

\begin{figure}[H]
  \begin{equation*}
    \begin{aligned}
      f \: (p_{11} \dots p_{1n}) = e_1 \\
      \dots                            \\
      f \: (p_{m1} \dots p_{mn}) = e_m \\
    \end{aligned}
  \end{equation*}
  \caption{A Function with $m$ Clauses}
\end{figure}

To type-check a function, we type-check all its clauses.

To type-check a clause, we first type-check all its patterns (in $Pattern.hs$), yielding an environment for the free variables in the right hand side $e$. After that, we check the right hand side.

\subsubsection{Type Checking of Patterns}
\label{sec:typeCheckPattern}

Recall that a pattern is either a \emph{variable} pattern, a \emph{constructor} pattern, or an \emph{inaccessible} pattern. A constructor pattern is parameterised by a \emph{name} and a \emph{list of patterns}. Therefore, a constructor pattern may have variable or inaccessible patterns as its parameter.

We type-check the patterns in two phases. In the first phase, we skip \emph{inaccessible} patterns, and only check the accessible part of the patterns, i.e., the variable and the constructor patterns.

The variable and inaccessible patterns are represented by fresh \emph{flexible} generic values. These flexible generic values are instantiated to concrete values when checking constructor patterns.

In the second phase, we verify that the expressions of the inaccessible patterns equal to those instantiated in the first phase.

\textbf{The first phase}, done by the function $\boldsymbol{checkPattern}$, calls the following functions:

\begin{enumerate}
  \item $\boldsymbol{patternToVal}$: Convert patterns to values. Variable patterns and inaccessible patterns are converted to \emph{flexible generic values}.
  \item $\boldsymbol{inst}$: Instantiate the flexible generic values (from step one) of constructor patterns to values by \emph{unification}.
\end{enumerate}

\begin{figure}[H]
  \begin{equation*}
    \begin{aligned}
      patternToVal \: k \: p             & = & v \textrm{ where } p2v \: k \: p = (v,k')                           \\
      \textrm{ and }                     &   &                                                                     \\
      p2v \: k \: x                      & = & (k,k+1)                                                             \\
      p2v \: k \: (c \: ())              & = & (c,k)                                                               \\
      p2v \: k \: (c \: (p_1 \dots p_n)) & = & (c \: (v_1 \dots v_n),k')                                           \\
                                         &   & \textrm{ where } (v_1 \dots v_n,k') = ps2vs \: k \: (p_1 \dots p_n) \\
      p2v \: k \: \underline{e}          & = & (k,k+1)                                                             \\
      \textrm{ and }                     &   &                                                                     \\
      ps2vs \: k \: ()                   & = & ((),k)                                                              \\
      ps2vs \: k \: (p_1 \dots p_n)      & = & (k'',(v_1,v_2,\dots,v_n))                                           \\
                                         &   & \textrm{ where } (v_1,k') = p2v \: k \: p \: \land                  \\
                                         &   & (v_2 \dots v_n,k'') = ps2vs \: k' \: (p_2 \dots p_n)                \\
    \end{aligned}
  \end{equation*}
  \caption{Converting Patterns to Values}
\end{figure}

For variable patterns ($x$) and inaccessible patterns (\underline{$e$}), they are converted to flexible generic values in $\boldsymbol{patternToVal}$. The parameter of a constructor pattern may include these patterns that are converted to flexible generic values. In $\boldsymbol{patternToVal}$, a constructor pattern with a non-empty list of patterns as its parameter is converted to an application of these values.

$\boldsymbol{inst}$ is the function that instantiates these flexible generic values to values. That is, $inst$ returns a list of mapping of generic values to values, which is called a \emph{substitution} or $\sigma$.

The application of a substitution on values ($\boldsymbol{substVal}$) and environment ($\boldsymbol{substEnv}$) are defined as follows:

\begin{figure}[H]
  \begin{equation*}
    \begin{aligned}
      substVal \: \sigma \: k                            & = & v \textrm{ if } (k,v) \in \sigma                                                           \\
      substVal \: \sigma \: (v \: (v_1 \dots v_n))       & = & (substVal \: \sigma \: v) \: (substVal \: \sigma \: v_1) \dots (substVal \: \sigma \: v_n) \\
      substVal \: \sigma \: (Pi \: x \: v_A \: B^{\rho}) & = & Pi \: x \: v_A \: B^{(substVal \: \sigma \: \rho)}                                         \\
      substVal \: \sigma \: (Lam \: x \: e^{\rho})       & = & Lam \: x \: e^{(substVal \: \sigma \: \rho)}                                               \\
      substVal \: \sigma \: v                            & = & v \textrm{ otherwise}                                                                      \\
      \\
      substEnv \: \sigma \: ()                           & = & ()                                                                                         \\
      substEnv \: \sigma \: ((x,v):\rho)                 & = & (x,(substVal \: \sigma \: v)) : (substEnv \: \sigma \: \rho)                               \\
    \end{aligned}
  \end{equation*}
  \caption{Application of a Substitution}
\end{figure}

$substVal$ returns a value while $substEnv$ returns an environment.

The composition of two substitutions ($\boldsymbol{compSubst}$) merges two substitutions into one, meaning that

\[
  compSubst \: \sigma_1 \: (substVal \: \sigma_2 \: v) = substVal \: \sigma_2 \: (substVal \: \sigma_1 \: v).
\]

\begin{figure}[H]
  \begin{equation*}
    \begin{aligned}
      compSubst \: ((k_1,v_1) \dots (k_n,v_n)) \: \sigma_2 & = & ((k_1, (substVal \: \sigma_2 \: v_1) \dots (k_n,(substVal \: \sigma_2 \: v_n)) \sigma_2) \\
    \end{aligned}
  \end{equation*}
  \caption{Composition of Two Substitutions}
\end{figure}

Assume that the domains of the substitutions are disjoint and that there are no occurrences of the generic values of $\sigma_1$ in the values of the codomain of $\sigma_2$.

$inst$ also calls $\boldsymbol{nonOccur}$, the function that checks that an atomic value $a$ does not occur in a value $v$.

$\boldsymbol{nonOccur}$ takes 3 arguments:

\begin{enumerate}
  \item $k$, the next fresh generic value.
  \item $a$, the atomic value ($k, \star, c, f, D$) being checked.
  \item $v$, the value to check that $a$ is not in.
\end{enumerate}

$nonOccur$ returns a boolean: true if $a$ is not in $v$, false otherwise.

\begin{figure}[H]
  \begin{equation*}
    \begin{aligned}
      nonOccur \: k \: a \: (Pi \: x \: v_A \: B^{\rho}) & = & nonOccur \: k \: a \: v_A \: \land                   \\
                                                         &   & nonOccur \: (k+1) \: a \: (eval B^{\rho:(x,k)})      \\
      nonOccur \: k \: a \: (Lam \: x \: e^{\rho})       & = & nonOccur \: (k+1) \: a \: (eval \: e^{\rho:(x,k)})   \\
      nonOccur \: k \: a \: (v \: (v_1 \dots v_n))       & = & nonOccur \: k \: a \: v \: \land                     \\
                                                         &   & nonOccur \: k \: a \: v_j, \forall j \in {1 \dots n} \\
      nonOccur \: k \: a \: a'                           & = & a \neq a'                                            \\
    \end{aligned}
  \end{equation*}
  \caption{Non-occurrence of Atomic Value $a$, $nonOccur$}
  \label{fig:nonOccur}
\end{figure}

$inst$ takes 4 arguments:

\begin{enumerate}
  \item $k$, the next fresh generic value.
  \item a set of flexible generic values of inaccessible patterns.
  \item [(3,4)] the values to unify.
\end{enumerate}

\begin{figure}[H]
  \begin{equation*}
    \begin{aligned}
      inst \: k \: [k_1,\dots,k_n] \: k' \: v                                            & =              & (k', v) \textrm{ if } k' \in [k_1 \dots k_n]                                                                                                                      \\
      inst \: k \: [k_1,\dots,k_n] \: v \: k'                                            & =              & (k', v) \textrm{ if } k' \in [k_1 \dots k_n]                                                                                                                      \\
      inst \: k \: [k_1,\dots,k_n] \: (c \: (v_1 \dots v_n)) \: (c \: (w_1 \dots w_n))   & =              & instList \: k \: [k_1,\dots,k_n] \: (v_1 \dots v_n) \: (w_1 \dots w_n)                                                                                            \\
      inst \: k \: [k_1,\dots,k_n] \: (D \: (v_1 \dots v_n)) \: (D \: (w_1 \dots w_n))   & =              & instList \: k \: [k_1,\dots,k_n] \: (v_1 \dots v_n) \: (w_1 \dots w_n)                                                                                            \\
      inst \: k \: [k_1,\dots,k_n] \: (f \: (v_1 \dots v_n)) \: (f \: (w_1 \dots w_n))   & =              & instList \: k \: [k_1,\dots,k_n] \: (v_1 \dots v_n) \: (w_1 \dots w_n)                                                                                            \\
      inst \: k \: [k_1,\dots,k_n] \: v_1 \: v_2                                         & =              & () \textrm{ if } eqVal \: k \: v_1 \: v_2                                                                                                                         \\
      \\
      instList \: k \: [k_1,\dots,k_n] \: () \: ()                                       & =              & ()                                                                                                                                                                \\
      instList \: k \: [k_1,\dots,k_n] \: (v_1 \: v_2 \: \dots) \: (w_1 \: w_2 \: \dots) & =              & compSubst \: \sigma \: \sigma'                                                                                                                                    \\
                                                                                         & \textrm{where} &                                                                                                                                                                   \\
      \sigma = inst \: k \: [k_1,\dots,k_n] \: v_1 \: w_2 \: \land                       &                &                                                                                                                                                                   \\
      \noalign{$\sigma'                                                                             =               instList \: k [k_1,\dots,k_n] \: ((substVal \sigma \: v_2) \dots (substVal \sigma \: v_n)) \: ((substVal \sigma \: w_2) \dots (substVal \sigma \: w_n))$} \\
    \end{aligned}
  \end{equation*}
  \caption{Instantiation of Flexible Values by Unification, $inst$}
\end{figure}

\section{Strict Positivity Checks}
\label{sec:spos}

This section is about a property of the constructors of a data type declaration (see section \ref{sec:datatypes}). Following figure \ref{fig:datatype}'s notations, the strict positivity check ensures that the data type $D$ only \emph{occur strictly positively} in $\Delta_i$. In other words, $D$ must not \emph{occur} (see figure \ref{fig:nonOccur}) in the left hand side of any $\Delta_i$ of its own constructor $c_i$.

The strict positivity condition rules out declarations that contain infinite loops. For example,

\begin{equation*}
  \begin{aligned}
    data &  & Bad : \star                 \\
         &  & bad : (Bad \to Bad) \to Bad \\
  \end{aligned}
\end{equation*}

The data type $Bad$ has one constructor, $bad$, it takes one argument, $\Delta_1 = (Bad \to Bad)$.

One can see from $\Delta_1$ that the data type $Bad$ is an argument to the function that is an input of its own constructor. Thus, there is a negative occurrence of $Bad$ in the type of the argument of the constructor. That is, $D$ does not occur strictly positively in $\Delta_1$. Non strictly-positive declarations are rejected because one can write a non-terminating function using them. Consider the following Haskell code using $Bad$ as defined above:

\begin{lstlisting}[language=haskell]
  getFun :: Bad -> (Bad -> Bad)
  getFun (bad f) = f
  
  omega :: Bad -> Bad
  omega f = (getFun f) f
  
  loop :: Bad
  loop = omega (bad omega)
\end{lstlisting}

$bad$ takes in a function $f$ with the type $Bad \to Bad$. Thus $bad \: f$ has type $Bad$. Because the return type of $(getFun \: f)$ is $(Bad \to Bad)$, $(getFun \: f) \: f$ has the return type of $Bad$. All these functions have proper types. Allowing non-strictly positive constructors allows $loop$, which infinitely loops.

In short, the strict positivity check ensures that the data type to be defined cannot occur in its constructor's arguments' function domain or in an application.

\subsection{Positive Parameters}

When declaring a data type, one can declare the requirement of certain parameters being positive. To declare a parameter positive, the data type declaration has an additional $\boldsymbol{+}$ at the beginning of the parameter, i.e., $\boldsymbol{(+ \: p_i:P_i)}$ instead of $(p_i : P_i)$. The type checking ensures that the declared positive parameters occur in the constructors strictly positively.

For example, the data type $Fun$ below is rejected by the type checker because $A$ is declared positive but it is not positive in the constructor $fn$:

\begin{figure}[H]
  \begin{equation*}
    \begin{aligned}
      data &  & Fun \: (+ \: A : \star) : \star \\
           &  & fn : (A \to A ) \to Fun \: A    \\
    \end{aligned}
  \end{equation*}
\end{figure}


We add a positivity tag to distinguish between strictly positive parameters and non-strictly positive parameters:

\begin{figure}[H]
  \begin{equation*}
    \begin{aligned}
      Pos &  & = &  & SPos  &  & \textrm{strictly positive}     \\
          &  & | &  & NSpos &  & \textrm{non-strictly positive} \\
    \end{aligned}
  \end{equation*}
\end{figure}

The set of positive parameter indices of a data type $D$ is represented by $\boldsymbol{pos(D)}$.

The information of whether a data type's parameters are strictly positive is stored in the signature for the check. The mapping for data type in the signature is now:

\begin{figure}[H]
  \begin{equation*}
    \begin{aligned}
      \Sigma D & : & (\textrm{value, Pos, integer}) \\
    \end{aligned}
  \end{equation*}
  \caption{Types of Data Type Mapping in a Signature}
\end{figure}

\subsection{Strictly Positive Occurrence}

To check that a data type occurs in its constructor arguments strictly positively, we only need to check its constructors that are functions. The data type cannot occur in constructors that don't take arguments. $sposConstructor$ checks that a constructor is strictly positive. It takes 4 arguments:

\begin{enumerate}
  \item the name of the data type being checked.
  \item the next fresh generic value.
  \item the list of positivity tag.
  \item the type of the constructor being checked.
\end{enumerate}

If the constructor is strictly positive, then $sposConstructor$ returns $unit$. Otherwise, an error occurs.

\begin{figure}[H]
  \begin{equation*}
    \begin{aligned}
      sposConstructor \: n \: k \: [pos] \: (Pi \: x \: v_A \: B^{\rho}) & = & spos \: k \: D \: v_A \: \land                                     \\
                                                                         &   & spos \: k \: j \: v_A, \: \forall j \in pos(D) \: \land            \\
                                                                         &   & sposConstructor \: n \: (k+1) \: [pos] \: (eval \: B^{\rho:(x,k)}) \\
    \end{aligned}
  \end{equation*}
  \caption{Strict Positivity Test for a Constructor}
\end{figure}

$sposConstructor$ calls $\boldsymbol{spos}$, which checks for strictly positive occurrence, and takes 3 arguments:

\begin{enumerate}
  \item the next fresh generic value.
  \item $a$ the value to check the occurrence of.
  \item the value that $a$ may occur in.
\end{enumerate}

$spos$ returns a boolean: if the data type occurs strictly positively, it returns true. Otherwise, it returns false.

\begin{figure}[H]
  \begin{equation*}
    \begin{aligned}
      spos \: k \: D \: (Pi \: x \: v_A \: B^{\rho}) & = & nonOccur \: k \: D \: v_A \: \land                                             \\
                                                     &   & spos \: (k+1) \: D \: (eval \: B^{\rho:(x,k)})                                 \\
      spos \: k \: D \: (Lam \: x \: e^{\rho})       & = & spos \: (k+1) \: D \: (eval \: e^{\rho:(x,k)})                                 \\
      spos \: k \: D \: (D \: v_1 \dots v_m)         & = & nonOccur \: k \: D \: v_j, \forall j \in \{1 \dots m\}, \notin pos(D) \: \land \\
                                                     &   & spos \: c \: D \: v_j, \forall j \in pos(D)                                    \\
      spos \: k \: D \: (v \: v_1 \dots v_n)         & = & spos \: k \: D \: v \: \land                                                   \\
                                                     &   & nonOccur \: k \: D \: v_j \forall j \in \{1 \dots n\}                          \\
      spos \: k \: D \: a                            & = & True                                                                           \\
    \end{aligned}
  \end{equation*}
  \caption{Strictly Positive Occurrence}
\end{figure}

$spos$ calls $nonOccur$ which is a stronger requirement than $spos$. $nonOccur$ checks that $D$ does not occur in the term at all while $spos$ checks that $D$ occurs in the term strictly positively.

We need to check for non-occurrence in certain circumstances in to determine strict positivity. For example, for a constructor which takes a function as an argument, $D$ has to not occur in the argument of that function.

$sposConstructor$ is called by $typeCheckConstructor$, which is called by $typeCheckDeclaration$ on data type declarations. Every constructor of the data type is checked by $sposConstructor$. If any constructor is not strictly positive, an error occurs during type checking.
\section{Termination Checks}
\label{sec:termination}

The type checker as is catches invalid expressions (by type checking) and data declarations that enable non-terminating functions (by checking strict positivity). However, one can still declare functions that infinitely loop. For example, the function declaration

\begin{lstlisting}[language=haskell]
  foo :: Nat -> Nat
  foo x = foo x
\end{lstlisting}

passes the type checker, although $foo$ is not well defined.

The termination checker (see $\boldsymbol{Termination.hs}$) aims to alleviate the problem in two ways:

\begin{enumerate}
  \item Analyzing the syntax of the program using the size-change principle (section \ref{sec:synTermination}).
  \item Analyzing the type validity using \emph{sized types} (section \ref{sec:typeTermination}).
\end{enumerate}

\subsection{Syntactic Checks}
\label{sec:synTermination}

The size-change principle states that \emph{a program terminates on all inputs if every infinite call sequence would cause an infinite descent in some data values.}

In order to determine whether a program terminates, we need to know how the size of the \emph{semantical} values change during evaluation of a recursion. If the expression of the arguments from a recursive call ($e$) is \emph{semantically smaller} than the patterns on the right hand side ($p$), then it passes this part of the termination checker.

\subsubsection{Order}

An \emph{order} is a set that denotes the possible results when comparing an expression $e$ to a pattern $p$.

$e$ is either

\begin{enumerate}
  \item \emph{smaller than} $p$ ($\boldsymbol{<}$),
  \item \emph{smaller than or equal to} $p$ ($\boldsymbol{\leq}$),
  \item or its relation to $p$ is \emph{unknown} ($\boldsymbol{?}$).
\end{enumerate}


\subsection{Type-based Checks}
\label{sec:typeTermination}

\section{Pattern Matching Coverage Checks}
\label{sec:pattern}

\end{document}
