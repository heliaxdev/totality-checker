Function definitions with pattern matching may involve clauses that are unreachable or redundant. The patterns may also not be exhaustive. We want to warn users of all these cases. We implement a pattern matching coverage check that checks for all these cases. See the $\boldsymbol{src/Coverage}$ directory for the implementation.

\subsection{Split Trees}

After the patterns are type checked (as in section \ref{sec:typeCheckPattern}), we transform the clauses into a \textit{split tree}. See $\boldsymbol{SplitTree.hs}$.

Each split tree is associated with either 

\begin{enumerate}
  \item a constructor ($SplitCon$)
  \item a literal ($SplitLit$)
  \item a catchall branch ($SplitCatchall$)
\end{enumerate}

A $SplitTag$ is added to each split tree to include this information. 

Each node of the split tree is either at an end ($SplittingDone$) or there is another split ($SplitAt$) of type $SplitTrees' a$, which is a list of constructor names and split trees mapping. It has sub-split trees for each of the constructors.

\subsection{Coverage Algorithm}

This algorithm works on function definitions with overlapping clauses. When two clauses overlap, the top most clause takes priority.

Given the list of clauses provided by the user, we compute a single \textit{covering} from a sequence of exhaustive but possibly overlapping clauses.

To compute a covering, we first compute \textit{an elementary covering}.

\subsubsection{Elementary and General Covering}

For full details see \textit{Pattern Matching with Dependent Types} by Thierry Coquand. My notations are a hybrid of Coquand and Norell's.

An \textit{elementary covering} of a common context $\Delta = x_1 : A_1,...x_n : A_n$ is a system of context mappings $S_1 : \Gamma_1 \to \Delta, ..., S_m : \Gamma_m \to \Delta$ iff there exists an index $\boldsymbol{i < n}$ such that

\begin{enumerate}
  \item all terms $x_iS_j : A_iS_j(\Delta_j)$, for $j \leq m$, are in constructor form and
  \item if $S : \Gamma \to \Delta$ is a context mapping such that $x_iS$ is in constructor form, then there exists one and only one $j \leq m$ and $T : \Gamma \to \Gamma_j$ such that $S=T;S_j$.
  
\end{enumerate}

For example, the elementary covering of the context $\Delta = x : N$ are the context mappings

\begin{itemize}
  \item $\{x := 0\} \to \Delta$
  \item $\{x := succ \ y \} : (y : N) \to \Delta$
\end{itemize}

A \textit{covering} of a common context $\Delta$ is a system of context mappings $S_i : \Delta_i \to \Delta$ if

\begin{enumerate}
  \item the identity interpretation $\Delta \to \Delta$ is a covering of $\Delta$ and
  \item if $S : \Gamma \to \Delta$ is a context mapping such that $x_iS$ is in constructor form, then there exists one and only one $j \leq m$ and $T : \Gamma \to \Gamma_j$ such that $S = T;S_j$. 
\end{enumerate}

For example, the following context mappings are a covering of the context $\boldsymbol{\Delta = x : N, y : N}$

\begin{itemize}
  \item $\{x := 0, y\} : (y : N) \to \Delta$,
  \item $\{x := succ(x_1), y:= 0\} : (x_1 : N) \to \Delta$ and
  \item $\{x := succ(x_1), y:= succ(y_1)\} : (x_1 : N, y_1 : N) \to \Delta$
\end{itemize}

A \textit{neighbourhood} of a context is any context mapping that is part of a covering of this context. For example, 

\[
  \{x := 0, y\} : (y : N) \to \Delta
\] is a neighbourhood of context $\Delta$ in the above example.

\subsubsection{Building a Covering (See $\boldsymbol{Coverage/Match.hs}$ and $\boldsymbol{Coverage.hs}$)} 

The algorithm starts with matching the initial context mapping (neighbourhood) with one of the clauses. We start the matching from the first clause. When there are terms that causes the matching to be indecisive (we call these terms \textit{blockers}), we perform an operation called \textit{splitting the context along $i$}, given $i$ is a blocker. As we split the context we yield neighbourhoods which we continue to match with the clauses. We perform context splitting recursively until the current neighbourhood matches one of the original clauses. When matching is successful, we return the covering. If the current neighbourhood fails to match all the given clauses, the clauses are not exhaustive and we terminate with coverage failure.

For example, for the following function:

\begin{lstlisting}[language=haskell]
  plus :: Nat -> Nat -> Nat
  plus x zero = x
  plus zero y = y
  plus (succ x) (succ y) = succ (x plus y)
\end{lstlisting}

We start out with the context mapping/neighbourhood

\[
  \{x,y\} : (x \ y : Nat) \to (x \ y : Nat)
\]

Because none of the clauses has $x$, $y$ as the patterns, we do not match any clause. Hence, we split the context along the blocker $y$. ($x$ successfully matches with $x$, so it's not a blocker). 
 
To split the context along $y$, we list all the constructors of type $Nat$ and perform unification. This generates a set of new context mappings where $y$ has been instantiated with an application of each of the constructors to fresh variables. In the example, we get the following neighbourhoods:

\begin{figure}[H]
  \begin{equation*}
    \begin{aligned}
      \delta_1 & = & \{x, y := zero\} : (x : Nat) \to (x \ y : Nat) \\
      \delta_2 & = & \{x, y := succ \ y\} : (x \ y : Nat) \to (x \ y : Nat) \\
    \end{aligned}
  \end{equation*}
\end{figure}

Because $\delta_1$ matches the first clause, we can return the first clause as the covering clause. Now we need to match $\delta_2$.

$\delta_2$ fails to match on the first clause so we move on to the second clause which is inconclusive (the variable $x$ is matching against the constructor $zero$). We move on to the third clause and we have the same inconclusive result. The blocker is now $x$ so we split the context along $x$, and yield the following neighbourhoods:

\begin{figure}[H]
  \begin{equation*}
    \begin{aligned}
      \delta_3 & = & \{x := zero, y := succ \ y\} : (y : Nat) \to (x \ y : Nat) \\
      \delta_4 & = & \{x := succ \ x; y := succ \ y\} : (x \ y : Nat) \to (x \ y : Nat) \\
    \end{aligned}
  \end{equation*}
\end{figure}

Now $\delta_3$ matches the second clause with the substitution $y = succ \ y$. We return the second clause as the covering clause. $\delta_4$ matches the third clause and we have finished building the covering from all neighbourhoods. The covering clauses are all three clauses.

Consider the following example with dependent types:

\begin{lstlisting}[language=haskell]
  typeOfX :: (x :: Nat) -> Type
  typeOfX zero = Nat
  typeOfX (succ _) = Bool
  
  plus2 :: (x :: Nat) -> (y :: typeOfX x) -> Nat
  plus2 zero y = y
  plus2 (succ z) True = z
  plus2 (succ z) False = succ (succ z)
\end{lstlisting}

To check the coverage of $plus2$, we start with the neighbourhood 

\[
  \{x,y\} : (x : Nat)(y : typeOfX \ x) \to (x : Nat)(y : typeOfX \ x)
\]

Matching the neighbourhood with the first clause, we see that $x$ is a blocker, so we split the context along $x$ and yield the following neighbourhoods:

\begin{figure}[H]
  \begin{equation*}
    \begin{aligned}
      \delta_1 & = & \{x := zero, y\} : (y : typeOfX \ zero) \to (x : Nat)(y : typeOfX \ x) \\
      \delta_2 & = & \{x := succ \ x, y\} : (x : Nat)(y : typeOfX (succ \ x)) \to (x : Nat)(y : typeOfX x) \\
    \end{aligned}
  \end{equation*}
\end{figure}

Matching $\delta_1$ with the first clause is successful. We can return the first clause. 

Matching $\delta_2$ with the first clause fails because $zero$ and $succ$ are different constructors and cannot be unified. Moving on to matching it with the second clause.

The second and third clauses are both inconclusive matches because of blocker $y$. Splitting the context along $y$.

To split the context along $y$, $y$ needs to be of type $small \ type$ such that we can list all of the constructors of that type. Because $y : typeOfX (succ \ x)$ and $typeOfX (succ \ x) = Bool$, we yield the following neighbourhoods:

\begin{figure}[H]
  \begin{equation*}
    \begin{aligned}
      \delta_3 & = & \{x := succ \ x, y := True\} : (x : Nat) \to (x : Nat)(y : typeOfX \ x) \\
      \delta_4 & = & \{x := succ \ x, y := False\} : (x: Nat) \to (x : Nat)(y : typeOfX \ x) \\
    \end{aligned}
  \end{equation*}
\end{figure}

With the substitution $x = z$, $\delta_3$ matches the second clause and $\delta_4$ matches the third clause. All neighbourhoods are matched with all clauses returned. The coverage check confirms that the function clauses covers all cases.

