The size-change principle states that \emph{a program terminates on all inputs if every infinite call sequence would cause an infinite descent in some data values.}

In order to determine whether a program terminates, we need to know how the size of the \emph{semantical} values change during evaluation of a recursion. If the expression of the arguments from a recursive call ($e$) is \emph{semantically smaller} than the patterns on the right hand side ($p$), then it passes this part of the termination checker.

\subsubsection{Order}

An \emph{order} is a set that denotes the possible results when comparing an expression $e$ to a pattern $p$.

$e$ is either

\begin{enumerate}
  \item \emph{smaller than} $p$ ($\boldsymbol{<}$),
  \item \emph{smaller than or equal to} $p$ ($\boldsymbol{\leq}$),
  \item or its relation to $p$ is \emph{unknown} ($\boldsymbol{?}$).
\end{enumerate}

The order is determined by the following axioms:

\begin{enumerate}
  \item $x < c \: (p_1 \dots p_n), \: \forall x \in {p_1, \dots p_n}$ and when $c$ is an inductive constructor.
  \item $f \: (e_1 \dots e_n) \leq f$
\end{enumerate}

